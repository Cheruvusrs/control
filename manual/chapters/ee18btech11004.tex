\begin{enumerate}[label=\thesubsection.\arabic*.,ref=\thesubsection.\theenumi]
\numberwithin{equation}{enumi}
\item State the general model of a state space system specifying the dimensions of the matrices and vectors.
\\
\solution The model is given by 
\begin{align}
\dot{\vec{x}}(t)&=\vec{A}\vec{x}(t)+\vec{B}\vec{u}(t) \\
 \vec{y}(t)&=\vec{C}\vec{x}(t)+\vec{D} \vec{u}(t)
\end{align}
%
with parameters listed in Table \ref{table:ee18btech11004}.
%
\begin{table}[!ht]
\centering
\begin{enumerate}[label=\thesection.\arabic*.,ref=\thesection.\theenumi]
\numberwithin{equation}{enumi}
\item State the general model of a state space system specifying the dimensions of the matrices and vectors.
\\
\solution The model is given by 
\begin{align}
\dot{\vec{x}}(t)&=\vec{A}\vec{x}(t)+\vec{B}\vec{u}(t) \\
 \vec{y}(t)&=\vec{C}\vec{x}(t)+\vec{D} \vec{u}(t)
\end{align}
%
with parameters listed in Table \ref{table:ee18btech11004}.
%
\begin{table}[!ht]
\centering
\begin{enumerate}[label=\thesection.\arabic*.,ref=\thesection.\theenumi]
\numberwithin{equation}{enumi}
\item State the general model of a state space system specifying the dimensions of the matrices and vectors.
\\
\solution The model is given by 
\begin{align}
\dot{\vec{x}}(t)&=\vec{A}\vec{x}(t)+\vec{B}\vec{u}(t) \\
 \vec{y}(t)&=\vec{C}\vec{x}(t)+\vec{D} \vec{u}(t)
\end{align}
%
with parameters listed in Table \ref{table:ee18btech11004}.
%
\begin{table}[!ht]
\centering
\begin{enumerate}[label=\thesection.\arabic*.,ref=\thesection.\theenumi]
\numberwithin{equation}{enumi}
\item State the general model of a state space system specifying the dimensions of the matrices and vectors.
\\
\solution The model is given by 
\begin{align}
\dot{\vec{x}}(t)&=\vec{A}\vec{x}(t)+\vec{B}\vec{u}(t) \\
 \vec{y}(t)&=\vec{C}\vec{x}(t)+\vec{D} \vec{u}(t)
\end{align}
%
with parameters listed in Table \ref{table:ee18btech11004}.
%
\begin{table}[!ht]
\centering
\input{./tables/ee18btech11004.tex}
\caption{}
\label{table:ee18btech11004}
\end{table}
\item Find the transfer function $\vec{H}(s)$ for the general system.
\\
\solution 
Taking Laplace transform on both sides we have the following equations
\begin{align}
s\vec{I}X(s)-x(0)= \vec{A}X(s)+ \vec{B}U(s)\\
(s\vec{I}-\vec{A})X(s)= \vec{B}U(s)+ x(0)\\
X(s)={(s\vec{I}-\vec{A})^{-1}}\vec{B} U(s)+ (s\vec{I}-\vec{A})^{-1}x(0)
\label{eq:x_init}
\end{align}
and
\begin{align}
Y(s)= \vec{C}X(s)+D\vec{I}U(s)
\end{align}
Substituting from \eqref{eq:x_init} in the above,
%
\begin{multline}
Y(s)=( \vec{C}{(s\vec{I}-\vec{A})^{-1}}\vec{B}+D\vec{I}) U(s) 
\\
+ \vec{C}(s\vec{I}-\vec{A})^{-1}x(0)
\end{multline}
%
\item Find $H(s)$ for a SISO (single input single output) system.
\\
\solution
\begin{align}
H(s)= {\frac{Y(s)}{U(s)}}= C{(sI-A)^{-1}}B+DI
\end{align}

\item Given 
\begin{align}
H(s)&=\frac{1}{s^3+3s^2+2s+1}
\\
D&=0
\\
\vec{B}&= \myvec{0\\0\\1}
\end{align}
%
 find $\vec{A}$ and $\vec{C}$ such that the state-space realization is in {\em controllable canonical form}.
\\
\solution 
\begin{align} 
\because {\frac{Y(s)}{U(s)}}= \frac{Y(s)}{V(s)} \times \frac{V(s)}{U(s)},
\end{align}
letting
\begin{align}
 {\frac{Y(s)}{V(s)}}= 1, 
\end{align}
results in 
\begin{align}
{\frac{U(s)}{V(s)}}={s^3 + 3s^2+2s + 1}
\end{align}

giving
\begin{align}
U(s)= s^3 V(s) + 3s^2 V(s)+2sV(s) + V(s)
\end{align}

so equation 0.1.13 can be written as
\begin{align}
\myvec{sV(s)\\s^2V(s)\\s^3V(s)}
=
\myvec{0&1&0\\0&0&1\\-1&-2&-3}\myvec{V(s)\\s(s)\\s^2V(s)}
+
\myvec{0\\0\\1}  U
\end{align}
So 
\begin{align}
\vec{A}=\myvec{0&1&0\\0&0&1\\-1&-2&-3}
\end{align}

\begin{align}
Y=X_{1}(s)
=\myvec{1&0&0} \myvec{V(s)\\sV(s)\\s^2V(s)}
\end{align}
\begin{align}
\vec{C}=\myvec{1&0&0}
\end{align}

\item Obtain $\vec{A}$ and $\vec{C}$ so that the state-space realization in in {\em observable canonical form}.
\\
\solution  Given that
\begin{align}
H(s)&=\frac{1}{s^3+3s^2+2s+1}
\end{align}
\begin{align}
\frac{Y(s)}{U(s)}=\frac{1}{s^3+3s^2+2s+1} \\
Y(s) \times (s^3+3s^2+2s+1) = U(s)
\end{align}
\begin{align}
s^3Y(s)+3s^2Y(s)+2sY(s)+Y(s)=U(s)\\
s^3Y(s)=U(s)-3s^2Y(s)-2sY(s)-Y(s)\\
Y(s)=-3s^{-1}Y(s)-2s^{-2}Y(s)+s^{-3}(U(s)-Y(s))
\end{align}
\\ let $Y=aU+X_{1}$
\\ by comparing with equation 1.5.6 we get a=0 and
\begin{align}
Y=X_{1}
\end{align}
inverse laplace transform of above equation is 
\begin{align}
y=x_{1}
\end{align}
so from above equation 1.5.6 and 1.5.7
\begin{align}
X_{1}=-3s^{-1}Y(s)-2s^{-2}Y(s)+s^{-3}(U(s)-Y(s))\\
sX_{1}=-3Y(s)-2s^{-1}Y(s)+s^{-2}(U(s)-Y(s)) 
\end{align}
inverse laplace transform of above equation 
\begin{align}
\dot{x_{1}}=-3y+x_{2}
\end{align} 
where
\begin{align}
X_{2}=-2s^{-1}Y(s)+s^{-2}(U(s)-Y(s))\\
sX_{2}=-2Y(s)+s^{-1}(U(s)-Y(s))
\end{align} 
inverse laplace transform of above equation 
\begin{align}
\dot{x_{2}}=-2y+x_{3}
\end{align}
where
\begin{align}
X_{3}=s^{-1}(U(s)-Y(s))\\
sX_{3}=U(s)-Y(s)
\end{align} 
inverse laplace transform of above equation 
\begin{align}
\dot{x_{3}}=u-y
\end{align}
so we get four equations which are
\begin{align}
y=x_{1}\\
\dot{x_{1}}=-3y+x_{2}\\
\dot{x_{2}}=-2y+x_{3}\\
\dot{x_{3}}=u-y
\end{align} 
sub $ y=x_{1}$ in 1.5.19,1.5.20,1.5.21 we get
\begin{align}
 y=x_{1}\\
\dot{x_{1}}=-3x_{1}+x_{2}\\
\dot{x_{2}}=-2x_{1}+x_{3}\\
\dot{x_{3}}=u-x_{1}
\end{align} 
so above equations can be written as
\begin{align}
\myvec{\dot{x_{1}}\\\dot{x_{2}}\\\dot{x_{3}})}
=
\myvec{-3&1&0\\-2&0&1\\-1&0&0}\myvec{x_{1}\\x_{2}\\x_{3}}
+
\myvec{0\\0\\1}  U
\end{align}
So 
\begin{align}
\vec{A}=\myvec{-3&1&0\\-2&0&1\\-1&0&0}
\end{align}
\begin{align}
y=x_{1}
=\myvec{1&0&0} \myvec{x_{1}\\x_{2}\\x_{3}}
\end{align}
\begin{align}
\vec{C}=\myvec{1&0&0}
\end{align}


\item Find the eigenvaues of $\vec{A}$ and the poles of $H(s)$ using a python code.
\\
\solution The following code 
%
\begin{lstlisting}
codes/ee18btech11004.py
\end{lstlisting}
gives the necessary values.  The roots are the same as the eigenvalues.
%
\item Theoretically, show that eigenvaues of $\vec{A}$ are the poles of  $H(s)$.
\solution 
\\ as we know tthat  the characteristic equation is det(sI-A) 
\\\begin{align}
\vec{sI-A}=
\myvec{s&0&0\\0&s&0\\0&0&s}
-
\myvec{0&1&0\\0&0&1\\-1&-2&-3}
=\myvec{s&-1&0\\0&s&-1\\1&2&s+3}
\end{align}
\\therfore
\begin{align}
det(sI-A)=s(s^2+3s+2)+1(1)=s^3+3s^2+2s+1
\end{align} 
\\so from equation 1.6.2 we can see that charcteristic equation is equal to the denominator of the transefer function
\end{enumerate}


\caption{}
\label{table:ee18btech11004}
\end{table}
\item Find the transfer function $\vec{H}(s)$ for the general system.
\\
\solution 
Taking Laplace transform on both sides we have the following equations
\begin{align}
s\vec{I}X(s)-x(0)= \vec{A}X(s)+ \vec{B}U(s)\\
(s\vec{I}-\vec{A})X(s)= \vec{B}U(s)+ x(0)\\
X(s)={(s\vec{I}-\vec{A})^{-1}}\vec{B} U(s)+ (s\vec{I}-\vec{A})^{-1}x(0)
\label{eq:x_init}
\end{align}
and
\begin{align}
Y(s)= \vec{C}X(s)+D\vec{I}U(s)
\end{align}
Substituting from \eqref{eq:x_init} in the above,
%
\begin{multline}
Y(s)=( \vec{C}{(s\vec{I}-\vec{A})^{-1}}\vec{B}+D\vec{I}) U(s) 
\\
+ \vec{C}(s\vec{I}-\vec{A})^{-1}x(0)
\end{multline}
%
\item Find $H(s)$ for a SISO (single input single output) system.
\\
\solution
\begin{align}
H(s)= {\frac{Y(s)}{U(s)}}= C{(sI-A)^{-1}}B+DI
\end{align}

\item Given 
\begin{align}
H(s)&=\frac{1}{s^3+3s^2+2s+1}
\\
D&=0
\\
\vec{B}&= \myvec{0\\0\\1}
\end{align}
%
 find $\vec{A}$ and $\vec{C}$ such that the state-space realization is in {\em controllable canonical form}.
\\
\solution 
\begin{align} 
\because {\frac{Y(s)}{U(s)}}= \frac{Y(s)}{V(s)} \times \frac{V(s)}{U(s)},
\end{align}
letting
\begin{align}
 {\frac{Y(s)}{V(s)}}= 1, 
\end{align}
results in 
\begin{align}
{\frac{U(s)}{V(s)}}={s^3 + 3s^2+2s + 1}
\end{align}

giving
\begin{align}
U(s)= s^3 V(s) + 3s^2 V(s)+2sV(s) + V(s)
\end{align}

so equation 0.1.13 can be written as
\begin{align}
\myvec{sV(s)\\s^2V(s)\\s^3V(s)}
=
\myvec{0&1&0\\0&0&1\\-1&-2&-3}\myvec{V(s)\\s(s)\\s^2V(s)}
+
\myvec{0\\0\\1}  U
\end{align}
So 
\begin{align}
\vec{A}=\myvec{0&1&0\\0&0&1\\-1&-2&-3}
\end{align}

\begin{align}
Y=X_{1}(s)
=\myvec{1&0&0} \myvec{V(s)\\sV(s)\\s^2V(s)}
\end{align}
\begin{align}
\vec{C}=\myvec{1&0&0}
\end{align}

\item Obtain $\vec{A}$ and $\vec{C}$ so that the state-space realization in in {\em observable canonical form}.
\\
\solution  Given that
\begin{align}
H(s)&=\frac{1}{s^3+3s^2+2s+1}
\end{align}
\begin{align}
\frac{Y(s)}{U(s)}=\frac{1}{s^3+3s^2+2s+1} \\
Y(s) \times (s^3+3s^2+2s+1) = U(s)
\end{align}
\begin{align}
s^3Y(s)+3s^2Y(s)+2sY(s)+Y(s)=U(s)\\
s^3Y(s)=U(s)-3s^2Y(s)-2sY(s)-Y(s)\\
Y(s)=-3s^{-1}Y(s)-2s^{-2}Y(s)+s^{-3}(U(s)-Y(s))
\end{align}
\\ let $Y=aU+X_{1}$
\\ by comparing with equation 1.5.6 we get a=0 and
\begin{align}
Y=X_{1}
\end{align}
inverse laplace transform of above equation is 
\begin{align}
y=x_{1}
\end{align}
so from above equation 1.5.6 and 1.5.7
\begin{align}
X_{1}=-3s^{-1}Y(s)-2s^{-2}Y(s)+s^{-3}(U(s)-Y(s))\\
sX_{1}=-3Y(s)-2s^{-1}Y(s)+s^{-2}(U(s)-Y(s)) 
\end{align}
inverse laplace transform of above equation 
\begin{align}
\dot{x_{1}}=-3y+x_{2}
\end{align} 
where
\begin{align}
X_{2}=-2s^{-1}Y(s)+s^{-2}(U(s)-Y(s))\\
sX_{2}=-2Y(s)+s^{-1}(U(s)-Y(s))
\end{align} 
inverse laplace transform of above equation 
\begin{align}
\dot{x_{2}}=-2y+x_{3}
\end{align}
where
\begin{align}
X_{3}=s^{-1}(U(s)-Y(s))\\
sX_{3}=U(s)-Y(s)
\end{align} 
inverse laplace transform of above equation 
\begin{align}
\dot{x_{3}}=u-y
\end{align}
so we get four equations which are
\begin{align}
y=x_{1}\\
\dot{x_{1}}=-3y+x_{2}\\
\dot{x_{2}}=-2y+x_{3}\\
\dot{x_{3}}=u-y
\end{align} 
sub $ y=x_{1}$ in 1.5.19,1.5.20,1.5.21 we get
\begin{align}
 y=x_{1}\\
\dot{x_{1}}=-3x_{1}+x_{2}\\
\dot{x_{2}}=-2x_{1}+x_{3}\\
\dot{x_{3}}=u-x_{1}
\end{align} 
so above equations can be written as
\begin{align}
\myvec{\dot{x_{1}}\\\dot{x_{2}}\\\dot{x_{3}})}
=
\myvec{-3&1&0\\-2&0&1\\-1&0&0}\myvec{x_{1}\\x_{2}\\x_{3}}
+
\myvec{0\\0\\1}  U
\end{align}
So 
\begin{align}
\vec{A}=\myvec{-3&1&0\\-2&0&1\\-1&0&0}
\end{align}
\begin{align}
y=x_{1}
=\myvec{1&0&0} \myvec{x_{1}\\x_{2}\\x_{3}}
\end{align}
\begin{align}
\vec{C}=\myvec{1&0&0}
\end{align}


\item Find the eigenvaues of $\vec{A}$ and the poles of $H(s)$ using a python code.
\\
\solution The following code 
%
\begin{lstlisting}
codes/ee18btech11004.py
\end{lstlisting}
gives the necessary values.  The roots are the same as the eigenvalues.
%
\item Theoretically, show that eigenvaues of $\vec{A}$ are the poles of  $H(s)$.
\solution 
\\ as we know tthat  the characteristic equation is det(sI-A) 
\\\begin{align}
\vec{sI-A}=
\myvec{s&0&0\\0&s&0\\0&0&s}
-
\myvec{0&1&0\\0&0&1\\-1&-2&-3}
=\myvec{s&-1&0\\0&s&-1\\1&2&s+3}
\end{align}
\\therfore
\begin{align}
det(sI-A)=s(s^2+3s+2)+1(1)=s^3+3s^2+2s+1
\end{align} 
\\so from equation 1.6.2 we can see that charcteristic equation is equal to the denominator of the transefer function
\end{enumerate}


\caption{}
\label{table:ee18btech11004}
\end{table}
\item Find the transfer function $\vec{H}(s)$ for the general system.
\\
\solution 
Taking Laplace transform on both sides we have the following equations
\begin{align}
s\vec{I}X(s)-x(0)= \vec{A}X(s)+ \vec{B}U(s)\\
(s\vec{I}-\vec{A})X(s)= \vec{B}U(s)+ x(0)\\
X(s)={(s\vec{I}-\vec{A})^{-1}}\vec{B} U(s)+ (s\vec{I}-\vec{A})^{-1}x(0)
\label{eq:x_init}
\end{align}
and
\begin{align}
Y(s)= \vec{C}X(s)+D\vec{I}U(s)
\end{align}
Substituting from \eqref{eq:x_init} in the above,
%
\begin{multline}
Y(s)=( \vec{C}{(s\vec{I}-\vec{A})^{-1}}\vec{B}+D\vec{I}) U(s) 
\\
+ \vec{C}(s\vec{I}-\vec{A})^{-1}x(0)
\end{multline}
%
\item Find $H(s)$ for a SISO (single input single output) system.
\\
\solution
\begin{align}
H(s)= {\frac{Y(s)}{U(s)}}= C{(sI-A)^{-1}}B+DI
\end{align}

\item Given 
\begin{align}
H(s)&=\frac{1}{s^3+3s^2+2s+1}
\\
D&=0
\\
\vec{B}&= \myvec{0\\0\\1}
\end{align}
%
 find $\vec{A}$ and $\vec{C}$ such that the state-space realization is in {\em controllable canonical form}.
\\
\solution 
\begin{align} 
\because {\frac{Y(s)}{U(s)}}= \frac{Y(s)}{V(s)} \times \frac{V(s)}{U(s)},
\end{align}
letting
\begin{align}
 {\frac{Y(s)}{V(s)}}= 1, 
\end{align}
results in 
\begin{align}
{\frac{U(s)}{V(s)}}={s^3 + 3s^2+2s + 1}
\end{align}

giving
\begin{align}
U(s)= s^3 V(s) + 3s^2 V(s)+2sV(s) + V(s)
\end{align}

so equation 0.1.13 can be written as
\begin{align}
\myvec{sV(s)\\s^2V(s)\\s^3V(s)}
=
\myvec{0&1&0\\0&0&1\\-1&-2&-3}\myvec{V(s)\\s(s)\\s^2V(s)}
+
\myvec{0\\0\\1}  U
\end{align}
So 
\begin{align}
\vec{A}=\myvec{0&1&0\\0&0&1\\-1&-2&-3}
\end{align}

\begin{align}
Y=X_{1}(s)
=\myvec{1&0&0} \myvec{V(s)\\sV(s)\\s^2V(s)}
\end{align}
\begin{align}
\vec{C}=\myvec{1&0&0}
\end{align}

\item Obtain $\vec{A}$ and $\vec{C}$ so that the state-space realization in in {\em observable canonical form}.
\\
\solution  Given that
\begin{align}
H(s)&=\frac{1}{s^3+3s^2+2s+1}
\end{align}
\begin{align}
\frac{Y(s)}{U(s)}=\frac{1}{s^3+3s^2+2s+1} \\
Y(s) \times (s^3+3s^2+2s+1) = U(s)
\end{align}
\begin{align}
s^3Y(s)+3s^2Y(s)+2sY(s)+Y(s)=U(s)\\
s^3Y(s)=U(s)-3s^2Y(s)-2sY(s)-Y(s)\\
Y(s)=-3s^{-1}Y(s)-2s^{-2}Y(s)+s^{-3}(U(s)-Y(s))
\end{align}
\\ let $Y=aU+X_{1}$
\\ by comparing with equation 1.5.6 we get a=0 and
\begin{align}
Y=X_{1}
\end{align}
inverse laplace transform of above equation is 
\begin{align}
y=x_{1}
\end{align}
so from above equation 1.5.6 and 1.5.7
\begin{align}
X_{1}=-3s^{-1}Y(s)-2s^{-2}Y(s)+s^{-3}(U(s)-Y(s))\\
sX_{1}=-3Y(s)-2s^{-1}Y(s)+s^{-2}(U(s)-Y(s)) 
\end{align}
inverse laplace transform of above equation 
\begin{align}
\dot{x_{1}}=-3y+x_{2}
\end{align} 
where
\begin{align}
X_{2}=-2s^{-1}Y(s)+s^{-2}(U(s)-Y(s))\\
sX_{2}=-2Y(s)+s^{-1}(U(s)-Y(s))
\end{align} 
inverse laplace transform of above equation 
\begin{align}
\dot{x_{2}}=-2y+x_{3}
\end{align}
where
\begin{align}
X_{3}=s^{-1}(U(s)-Y(s))\\
sX_{3}=U(s)-Y(s)
\end{align} 
inverse laplace transform of above equation 
\begin{align}
\dot{x_{3}}=u-y
\end{align}
so we get four equations which are
\begin{align}
y=x_{1}\\
\dot{x_{1}}=-3y+x_{2}\\
\dot{x_{2}}=-2y+x_{3}\\
\dot{x_{3}}=u-y
\end{align} 
sub $ y=x_{1}$ in 1.5.19,1.5.20,1.5.21 we get
\begin{align}
 y=x_{1}\\
\dot{x_{1}}=-3x_{1}+x_{2}\\
\dot{x_{2}}=-2x_{1}+x_{3}\\
\dot{x_{3}}=u-x_{1}
\end{align} 
so above equations can be written as
\begin{align}
\myvec{\dot{x_{1}}\\\dot{x_{2}}\\\dot{x_{3}})}
=
\myvec{-3&1&0\\-2&0&1\\-1&0&0}\myvec{x_{1}\\x_{2}\\x_{3}}
+
\myvec{0\\0\\1}  U
\end{align}
So 
\begin{align}
\vec{A}=\myvec{-3&1&0\\-2&0&1\\-1&0&0}
\end{align}
\begin{align}
y=x_{1}
=\myvec{1&0&0} \myvec{x_{1}\\x_{2}\\x_{3}}
\end{align}
\begin{align}
\vec{C}=\myvec{1&0&0}
\end{align}


\item Find the eigenvaues of $\vec{A}$ and the poles of $H(s)$ using a python code.
\\
\solution The following code 
%
\begin{lstlisting}
codes/ee18btech11004.py
\end{lstlisting}
gives the necessary values.  The roots are the same as the eigenvalues.
%
\item Theoretically, show that eigenvaues of $\vec{A}$ are the poles of  $H(s)$.
\solution 
\\ as we know tthat  the characteristic equation is det(sI-A) 
\\\begin{align}
\vec{sI-A}=
\myvec{s&0&0\\0&s&0\\0&0&s}
-
\myvec{0&1&0\\0&0&1\\-1&-2&-3}
=\myvec{s&-1&0\\0&s&-1\\1&2&s+3}
\end{align}
\\therfore
\begin{align}
det(sI-A)=s(s^2+3s+2)+1(1)=s^3+3s^2+2s+1
\end{align} 
\\so from equation 1.6.2 we can see that charcteristic equation is equal to the denominator of the transefer function
\end{enumerate}


\caption{}
\label{table:ee18btech11004}
\end{table}
\item Find the transfer function $\vec{H}(s)$ for the general system.
\\
\solution 
Taking Laplace transform on both sides we have the following equations
\begin{align}
s\vec{I}X(s)-x(0)= \vec{A}X(s)+ \vec{B}U(s)\\
(s\vec{I}-\vec{A})X(s)= \vec{B}U(s)+ x(0)\\
X(s)={(s\vec{I}-\vec{A})^{-1}}\vec{B} U(s)+ (s\vec{I}-\vec{A})^{-1}x(0)
\label{eq:x_init}
\end{align}
and
\begin{align}
Y(s)= \vec{C}X(s)+D\vec{I}U(s)
\end{align}
Substituting from \eqref{eq:x_init} in the above,
%
\begin{multline}
Y(s)=( \vec{C}{(s\vec{I}-\vec{A})^{-1}}\vec{B}+D\vec{I}) U(s) 
\\
+ \vec{C}(s\vec{I}-\vec{A})^{-1}x(0)
\end{multline}
%
\item Find $H(s)$ for a SISO (single input single output) system.
\\
\solution
\begin{align}
H(s)= {\frac{Y(s)}{U(s)}}= C{(sI-A)^{-1}}B+DI
\end{align}

\item Given 
\begin{align}
H(s)&=\frac{1}{s^3+3s^2+2s+1}
\\
D&=0
\\
\vec{B}&= \myvec{0\\0\\1}
\end{align}
%
 find $\vec{A}$ and $\vec{C}$ such that the state-space realization is in {\em controllable canonical form}.
\\
\solution 
\begin{align} 
\because {\frac{Y(s)}{U(s)}}= \frac{Y(s)}{V(s)} \times \frac{V(s)}{U(s)},
\end{align}
letting
\begin{align}
 {\frac{Y(s)}{V(s)}}= 1, 
\end{align}
results in 
\begin{align}
{\frac{U(s)}{V(s)}}={s^3 + 3s^2+2s + 1}
\end{align}

giving
\begin{align}
U(s)= s^3 V(s) + 3s^2 V(s)+2sV(s) + V(s)
\end{align}

so equation 0.1.13 can be written as
\begin{align}
\myvec{sV(s)\\s^2V(s)\\s^3V(s)}
=
\myvec{0&1&0\\0&0&1\\-1&-2&-3}\myvec{V(s)\\s(s)\\s^2V(s)}
+
\myvec{0\\0\\1}  U
\end{align}
So 
\begin{align}
\vec{A}=\myvec{0&1&0\\0&0&1\\-1&-2&-3}
\end{align}

\begin{align}
Y=X_{1}(s)
=\myvec{1&0&0} \myvec{V(s)\\sV(s)\\s^2V(s)}
\end{align}
\begin{align}
\vec{C}=\myvec{1&0&0}
\end{align}

\item Obtain $\vec{A}$ and $\vec{C}$ so that the state-space realization in in {\em observable canonical form}.
\\
\solution  Given that
\begin{align}
H(s)&=\frac{1}{s^3+3s^2+2s+1}
\end{align}
\begin{align}
\frac{Y(s)}{U(s)}=\frac{1}{s^3+3s^2+2s+1} \\
Y(s) \times (s^3+3s^2+2s+1) = U(s)
\end{align}
\begin{align}
s^3Y(s)+3s^2Y(s)+2sY(s)+Y(s)=U(s)\\
s^3Y(s)=U(s)-3s^2Y(s)-2sY(s)-Y(s)\\
Y(s)=-3s^{-1}Y(s)-2s^{-2}Y(s)+s^{-3}(U(s)-Y(s))
\end{align}
\\ let $Y=aU+X_{1}$
\\ by comparing with equation 1.5.6 we get a=0 and
\begin{align}
Y=X_{1}
\end{align}
inverse laplace transform of above equation is 
\begin{align}
y=x_{1}
\end{align}
so from above equation 1.5.6 and 1.5.7
\begin{align}
X_{1}=-3s^{-1}Y(s)-2s^{-2}Y(s)+s^{-3}(U(s)-Y(s))\\
sX_{1}=-3Y(s)-2s^{-1}Y(s)+s^{-2}(U(s)-Y(s)) 
\end{align}
inverse laplace transform of above equation 
\begin{align}
\dot{x_{1}}=-3y+x_{2}
\end{align} 
where
\begin{align}
X_{2}=-2s^{-1}Y(s)+s^{-2}(U(s)-Y(s))\\
sX_{2}=-2Y(s)+s^{-1}(U(s)-Y(s))
\end{align} 
inverse laplace transform of above equation 
\begin{align}
\dot{x_{2}}=-2y+x_{3}
\end{align}
where
\begin{align}
X_{3}=s^{-1}(U(s)-Y(s))\\
sX_{3}=U(s)-Y(s)
\end{align} 
inverse laplace transform of above equation 
\begin{align}
\dot{x_{3}}=u-y
\end{align}
so we get four equations which are
\begin{align}
y=x_{1}\\
\dot{x_{1}}=-3y+x_{2}\\
\dot{x_{2}}=-2y+x_{3}\\
\dot{x_{3}}=u-y
\end{align} 
sub $ y=x_{1}$ in 1.5.19,1.5.20,1.5.21 we get
\begin{align}
 y=x_{1}\\
\dot{x_{1}}=-3x_{1}+x_{2}\\
\dot{x_{2}}=-2x_{1}+x_{3}\\
\dot{x_{3}}=u-x_{1}
\end{align} 
so above equations can be written as
\begin{align}
\myvec{\dot{x_{1}}\\\dot{x_{2}}\\\dot{x_{3}})}
=
\myvec{-3&1&0\\-2&0&1\\-1&0&0}\myvec{x_{1}\\x_{2}\\x_{3}}
+
\myvec{0\\0\\1}  U
\end{align}
So 
\begin{align}
\vec{A}=\myvec{-3&1&0\\-2&0&1\\-1&0&0}
\end{align}
\begin{align}
y=x_{1}
=\myvec{1&0&0} \myvec{x_{1}\\x_{2}\\x_{3}}
\end{align}
\begin{align}
\vec{C}=\myvec{1&0&0}
\end{align}


\item Find the eigenvaues of $\vec{A}$ and the poles of $H(s)$ using a python code.
\\
\solution The following code 
%
\begin{lstlisting}
codes/ee18btech11004.py
\end{lstlisting}
gives the necessary values.  The roots are the same as the eigenvalues.
%
\item Theoretically, show that eigenvaues of $\vec{A}$ are the poles of  $H(s)$.
\solution 
\\ as we know tthat  the characteristic equation is det(sI-A) 
\\\begin{align}
\vec{sI-A}=
\myvec{s&0&0\\0&s&0\\0&0&s}
-
\myvec{0&1&0\\0&0&1\\-1&-2&-3}
=\myvec{s&-1&0\\0&s&-1\\1&2&s+3}
\end{align}
\\therfore
\begin{align}
det(sI-A)=s(s^2+3s+2)+1(1)=s^3+3s^2+2s+1
\end{align} 
\\so from equation 1.6.2 we can see that charcteristic equation is equal to the denominator of the transefer function
\end{enumerate}

