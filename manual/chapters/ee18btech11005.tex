\begin{enumerate}[label=\thesection.\arabic*.,ref=\thesection.\theenumi]
\numberwithin{equation}{enumi}

\item
Consider a unity feedback system as shown in Fig.  \ref{fig:ee18btech11005}, with an integral compensator $\frac{k}{s}$ and open-loop transfer function
\begin{align}
G(s) = \frac{1}{s^2+3s+2}
\end{align}
where k greater than 0. 
%
Find its closed loop transfer function.
\begin{figure}[!ht]
\tikzstyle{block} = [draw, fill=blue!20, rectangle, 
    minimum height=1cm, minimum width=1cm]
\tikzstyle{sum} = [draw, fill=blue!20, circle, node distance=1cm]
\tikzstyle{input} = [coordinate]
\tikzstyle{output} = [coordinate]
\tikzstyle{pinstyle} = [pin edge={to-,thin,black}]

% The block diagram code is probably more verbose than necessary
\begin{tikzpicture}[auto, node distance=2cm,>=latex']
    % We start by placing the blocks
    \node [input, name=input] {X(s)};
    \node [sum, right of=input] (sum) {};
    \node [block, right of=sum] (controller) {G(s)};
    \node [block, right of=controller] (system) {$k/s$};
    % We draw an edge between the controller and system block to 
    % calculate the coordinate u. We need it to place the measurement block. 
    \draw [->] (controller) -- node[name=u] {} (system);
    \node [output, right of=system] (output) {};
    \node [block, below of=u] (measurements) {1};

    % Once the nodes are placed, connecting them is easy. 
    \draw [draw,->] (input) -- node {$X(s)$} (sum);
    \draw [->] (sum) -- node {} (controller);
    \draw [->] (system) -- node [name=y] {$Y(s)$}(output);
    \draw [->] (y) |- (measurements);
    \draw [->] (measurements) -| node[pos=0.99] {$-$} 
        node [near end] {} (sum);
\end{tikzpicture}
\caption{}
\label{fig:ee18btech11005}
\end{figure}

\solution The transfer function for negative feedback is given by
\begin{align}
\frac{Y(s)}{X(s)} &= \frac{G(s)}{1+G(s)H(s)}
\end{align}
where H(s) = 1 for unity feedback system
and G(s) is net forward open loop gain
\begin{align}
G(s) &=  \sbrak{\frac{1}{s^2+3s+2}}\sbrak{\frac{k}{s}}
&= \frac{k}{s^3+3s^2+2s}
\end{align}
\item Find the {\em characteristic} equation for $G(s)$.

Characteristic equation is..,
\begin{align}
 1 + G(s)H(s) = 0 \label{eq:sec_order_op}
\\
=> 1 + \sbrak{\frac{k}{s^3+3s^2+2s}} = 0
\\
=> s^3+3s^2+2s+k = 0
\end{align}
\item Using the tabular method for the Routh hurwitz criterion, find $k > 0$ for which there are two poles of unity feedback system on j${\omega}$ axis.
%
\\
\solution 
This criterion is based on arranging the coefficients of characteristic equation into an array called Routh array.
For any characteristic equation q(s),
\begin{align}
q(s) = a_0s^n+a_1s^{n-1}+a_2s^{n-2}+.....+a_{n-1}s+a_n = 0
\end{align}
Routh array can be constructed as follows..,
 
\myvec{s^n\\s^{n-1}\\s^{n-2} \\ \vdots}
 \myvec{a_0 & a_2 & a_4 & \cdots \\
a_1 & a_3 & a_5 & \cdots  \\
b_1 & b_2 & b_3 & \cdots \\
\vdots & \vdots & \vdots & \ddots &\vdots 
 \cdots \\}
 \\
 where
 \begin{align}
 b_1 =\frac{ a_1a_2-a_0a_3}{a_1}  
 \\
 b_2 =\frac{ a_1a_4-a_0a_5}{a_1} 
 \\
 c_1=\frac{ b_1a_3-a_1b_2}{b_1} 
\\
 c_2=\frac{ b_1a_5-a_1b_3}{b_1}  
\end{align}
\\
For poles to lie on imaginary axis any one entire row of hurwitz matrix should be zero.
Constructing the routh array for the characteristic equation obtained in equation(\ref{eq:sec_order_op})
\begin{align}
 s^3+3s^2+2s+k = 0
\end{align}
%
\begin{align}
\myvec{s^3\\s^2\\s^1 \\ s^0}
\myvec{1 & 2 \\ 3 & k \\  \frac{6-k}{3} & 0\\ k & 0}
\end{align}
\\
For poles on j$\omega$ axis any one of the row should be zero.
\\
\begin{align}
\frac{6-k}{3} = 0 \hspace{5pt} or\hspace{5pt} k = 0
\end{align}
\\
But given k greater than 0 ...
\begin{align}
   6-k = 0\\
   k = 6
\end{align}
\item Repeat the above using the determinant method.
\item Verify your answer using a python code for both the determinant method as well as the tabular method.
\end{enumerate}
