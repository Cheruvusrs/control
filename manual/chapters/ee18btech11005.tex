\begin{enumerate}[label=\thesubsection.\arabic*.,ref=\thesubsection.\theenumi]
\numberwithin{equation}{enumi}

\item
Consider a unity feedback system as shown in Fig.  \ref{fig:ee18btech11005}, with an integral compensator $\frac{k}{s}$ and open-loop transfer function
\begin{align}
G(s) = \frac{1}{s^2+3s+2}
\end{align}
where k greater than 0. 
%
Find its closed loop transfer function.
\begin{figure}[!ht]
	\begin{center}
		
		\resizebox{\columnwidth}{!}{\begin{enumerate}[label=\thesection.\arabic*.,ref=\thesection.\theenumi]
\numberwithin{equation}{enumi}

\item
Consider a unity feedback system as shown in Fig.  \ref{fig:ee18btech11005}, with an integral compensator $\frac{k}{s}$ and open-loop transfer function
\begin{align}
G(s) = \frac{1}{s^2+3s+2}
\end{align}
where k greater than 0. 
%
Find its closed loop transfer function.
\begin{figure}[!ht]
	\begin{center}
		
		\resizebox{\columnwidth}{!}{\begin{enumerate}[label=\thesection.\arabic*.,ref=\thesection.\theenumi]
\numberwithin{equation}{enumi}

\item
Consider a unity feedback system as shown in Fig.  \ref{fig:ee18btech11005}, with an integral compensator $\frac{k}{s}$ and open-loop transfer function
\begin{align}
G(s) = \frac{1}{s^2+3s+2}
\end{align}
where k greater than 0. 
%
Find its closed loop transfer function.
\begin{figure}[!ht]
	\begin{center}
		
		\resizebox{\columnwidth}{!}{\begin{enumerate}[label=\thesection.\arabic*.,ref=\thesection.\theenumi]
\numberwithin{equation}{enumi}

\item
Consider a unity feedback system as shown in Fig.  \ref{fig:ee18btech11005}, with an integral compensator $\frac{k}{s}$ and open-loop transfer function
\begin{align}
G(s) = \frac{1}{s^2+3s+2}
\end{align}
where k greater than 0. 
%
Find its closed loop transfer function.
\begin{figure}[!ht]
	\begin{center}
		
		\resizebox{\columnwidth}{!}{\input{./figs/ee18btech11005.tex}}
	\end{center}
\caption{}
\label{fig:ee18btech11005}
\end{figure}

\solution $\because H(s) = 1$ in Fig.  \ref{fig:ee18btech11005}, due to unity feedback,   the transfer function is given by
\begin{align}
\frac{Y(s)}{X(s)} &= \frac{G(s)}{1+G(s)H(s)}
\\
\implies T(s) &= \frac{k}{s^3+3s^2+2s}
\end{align}
%
\item Find the {\em characteristic} equation for $G(s)$.
\\
\solution The characteristic equation is
\begin{align}
\label{eq:routh_char_eq}
 1 + G(s)H(s) &= 0 
\\
\implies 1 + \sbrak{\frac{k}{s^3+3s^2+2s}} &= 0
\\
\text{or, } s^3+3s^2+2s+k &= 0
\end{align}
\item Using the tabular method for the Routh hurwitz criterion, find $k > 0$ for which there are two poles of unity feedback system on j${\omega}$ axis.
%
\\
\solution 
This criterion is based on arranging the coefficients of characteristic equation into an array called Routh array.
For any characteristic equation 
\begin{multline}
q(s) = a_0s^n+a_1s^{n-1}+.....+a_{n-1}s+a_n = 0
\end{multline}
the Routh array can be constructed as 
 
\begin{align}
\mydet{s^n\\s^{n-1}\\s^{n-2} \\ \vdots}
 \mydet{a_0 & a_2 & a_4 & \cdots \\
a_1 & a_3 & a_5 & \cdots  \\
b_1 & b_2 & b_3 & \cdots \\
\vdots & \vdots & \vdots & \ddots &\vdots 
 \cdots \\}
\end{align}
%
 where
 \begin{align}
 b_1 =\frac{ a_1a_2-a_0a_3}{a_1}  
 \\
 b_2 =\frac{ a_1a_4-a_0a_5}{a_1} 
 \\
 c_1=\frac{ b_1a_3-a_1b_2}{b_1} 
\\
 c_2=\frac{ b_1a_5-a_1b_3}{b_1}  
\end{align}
For poles to lie on imaginary axis any one entire row of hurwitz matrix should be zero.
Constructing the routh array for the characteristic equation obtained in \ref{eq:routh_char_eq},
%
\begin{align}
 s^3+3s^2+2s+k = 0
\end{align}
%
\begin{align}
\mydet{s^3\\s^2\\s^1 \\ s^0}
\mydet{1 & 2 \\ 3 & k \\  \frac{6-k}{3} & 0\\ k & 0}
\end{align}
For poles on $\j \omega$ axis any one of the row should be zero.
%
\begin{align}
\therefore \frac{6-k}{3} &= 0 \text{ or } k = 0
\\
\implies k &= 6 \quad \because k > 0
\end{align}
\item Repeat the above using the determinant method.
\\
\solution The {\em Routh matrix} can be expressed as
\begin{align}
\vec{R} = \myvec{a_0 & a_2 & a_4 & \cdots \\
a_1 & a_3 & a_5 & \cdots  \\
 0 & a_0 & a_2\cdots \\
 0 & a_1 & a_3 \cdots\\
\vdots & \vdots & \vdots & \ddots &\vdots 
\cdots \\}
\end{align}
and the corresponding Routh determinants are
\begin{align}
D_1 &= |a_0|
\\
D_2 &= 
\mydet{
a_0 & a_2 
\\ 
a_1 & a_3
} 
\\
D_3 &=\mydet{
a_0 & a_2 & a_4 
\\ a_1 & a_3 & a_5 
\\ 0 & a_0 & a_2}
\\
\dots
\end{align}
If at least any one of the Determinents are zero then the poles lie on imaginary axes.  From \eqref{eq:routh_char_eq},
%
\begin{align}
D_1 &= 1 \ne 0
\\
D2 &= \mydet{
1 & 2 \\ 3 & k } 
&= k-6 =0 \implies k = 6
\end{align}
%
\item Verify your answer using a python code for both the determinant method as well as the tabular method.
\\
\solution 
The following code 
%
\begin{lstlisting}
codes/ee18btech11005/ee18btech11005.py
\end{lstlisting}
%
provides the necessary soution.
\begin{itemize}
\item  For the system to be stable all coefficients should lie on left half of s-plane. Because if any pole is in right half of s-plane then there will be a component in output that increases without bound,causing system to be unstable.
All the coefficients in the characteristic equation should be positive.This is necessary condition but not sufficient.Because it may have poles on right half of s plane.
Poles are the roots of the characteristic equation.
    \item A system is stable if all of its characteristic modes go to finite value as t goes to infinity.It is possible only if all the poles are on the left half of s plane.
    The characteristic equation should have negative roots only. So the first column should always be greater than zero.That means no sign changes.
    \item A system is unstable if its characteristic modes are not bounded. Then the characteristic equation will also have roots in the right side of s-plane.That means it has sign changes.
    \end{itemize}

\end{enumerate}


}
	\end{center}
\caption{}
\label{fig:ee18btech11005}
\end{figure}

\solution $\because H(s) = 1$ in Fig.  \ref{fig:ee18btech11005}, due to unity feedback,   the transfer function is given by
\begin{align}
\frac{Y(s)}{X(s)} &= \frac{G(s)}{1+G(s)H(s)}
\\
\implies T(s) &= \frac{k}{s^3+3s^2+2s}
\end{align}
%
\item Find the {\em characteristic} equation for $G(s)$.
\\
\solution The characteristic equation is
\begin{align}
\label{eq:routh_char_eq}
 1 + G(s)H(s) &= 0 
\\
\implies 1 + \sbrak{\frac{k}{s^3+3s^2+2s}} &= 0
\\
\text{or, } s^3+3s^2+2s+k &= 0
\end{align}
\item Using the tabular method for the Routh hurwitz criterion, find $k > 0$ for which there are two poles of unity feedback system on j${\omega}$ axis.
%
\\
\solution 
This criterion is based on arranging the coefficients of characteristic equation into an array called Routh array.
For any characteristic equation 
\begin{multline}
q(s) = a_0s^n+a_1s^{n-1}+.....+a_{n-1}s+a_n = 0
\end{multline}
the Routh array can be constructed as 
 
\begin{align}
\mydet{s^n\\s^{n-1}\\s^{n-2} \\ \vdots}
 \mydet{a_0 & a_2 & a_4 & \cdots \\
a_1 & a_3 & a_5 & \cdots  \\
b_1 & b_2 & b_3 & \cdots \\
\vdots & \vdots & \vdots & \ddots &\vdots 
 \cdots \\}
\end{align}
%
 where
 \begin{align}
 b_1 =\frac{ a_1a_2-a_0a_3}{a_1}  
 \\
 b_2 =\frac{ a_1a_4-a_0a_5}{a_1} 
 \\
 c_1=\frac{ b_1a_3-a_1b_2}{b_1} 
\\
 c_2=\frac{ b_1a_5-a_1b_3}{b_1}  
\end{align}
For poles to lie on imaginary axis any one entire row of hurwitz matrix should be zero.
Constructing the routh array for the characteristic equation obtained in \ref{eq:routh_char_eq},
%
\begin{align}
 s^3+3s^2+2s+k = 0
\end{align}
%
\begin{align}
\mydet{s^3\\s^2\\s^1 \\ s^0}
\mydet{1 & 2 \\ 3 & k \\  \frac{6-k}{3} & 0\\ k & 0}
\end{align}
For poles on $\j \omega$ axis any one of the row should be zero.
%
\begin{align}
\therefore \frac{6-k}{3} &= 0 \text{ or } k = 0
\\
\implies k &= 6 \quad \because k > 0
\end{align}
\item Repeat the above using the determinant method.
\\
\solution The {\em Routh matrix} can be expressed as
\begin{align}
\vec{R} = \myvec{a_0 & a_2 & a_4 & \cdots \\
a_1 & a_3 & a_5 & \cdots  \\
 0 & a_0 & a_2\cdots \\
 0 & a_1 & a_3 \cdots\\
\vdots & \vdots & \vdots & \ddots &\vdots 
\cdots \\}
\end{align}
and the corresponding Routh determinants are
\begin{align}
D_1 &= |a_0|
\\
D_2 &= 
\mydet{
a_0 & a_2 
\\ 
a_1 & a_3
} 
\\
D_3 &=\mydet{
a_0 & a_2 & a_4 
\\ a_1 & a_3 & a_5 
\\ 0 & a_0 & a_2}
\\
\dots
\end{align}
If at least any one of the Determinents are zero then the poles lie on imaginary axes.  From \eqref{eq:routh_char_eq},
%
\begin{align}
D_1 &= 1 \ne 0
\\
D2 &= \mydet{
1 & 2 \\ 3 & k } 
&= k-6 =0 \implies k = 6
\end{align}
%
\item Verify your answer using a python code for both the determinant method as well as the tabular method.
\\
\solution 
The following code 
%
\begin{lstlisting}
codes/ee18btech11005/ee18btech11005.py
\end{lstlisting}
%
provides the necessary soution.
\begin{itemize}
\item  For the system to be stable all coefficients should lie on left half of s-plane. Because if any pole is in right half of s-plane then there will be a component in output that increases without bound,causing system to be unstable.
All the coefficients in the characteristic equation should be positive.This is necessary condition but not sufficient.Because it may have poles on right half of s plane.
Poles are the roots of the characteristic equation.
    \item A system is stable if all of its characteristic modes go to finite value as t goes to infinity.It is possible only if all the poles are on the left half of s plane.
    The characteristic equation should have negative roots only. So the first column should always be greater than zero.That means no sign changes.
    \item A system is unstable if its characteristic modes are not bounded. Then the characteristic equation will also have roots in the right side of s-plane.That means it has sign changes.
    \end{itemize}

\end{enumerate}


}
	\end{center}
\caption{}
\label{fig:ee18btech11005}
\end{figure}

\solution $\because H(s) = 1$ in Fig.  \ref{fig:ee18btech11005}, due to unity feedback,   the transfer function is given by
\begin{align}
\frac{Y(s)}{X(s)} &= \frac{G(s)}{1+G(s)H(s)}
\\
\implies T(s) &= \frac{k}{s^3+3s^2+2s}
\end{align}
%
\item Find the {\em characteristic} equation for $G(s)$.
\\
\solution The characteristic equation is
\begin{align}
\label{eq:routh_char_eq}
 1 + G(s)H(s) &= 0 
\\
\implies 1 + \sbrak{\frac{k}{s^3+3s^2+2s}} &= 0
\\
\text{or, } s^3+3s^2+2s+k &= 0
\end{align}
\item Using the tabular method for the Routh hurwitz criterion, find $k > 0$ for which there are two poles of unity feedback system on j${\omega}$ axis.
%
\\
\solution 
This criterion is based on arranging the coefficients of characteristic equation into an array called Routh array.
For any characteristic equation 
\begin{multline}
q(s) = a_0s^n+a_1s^{n-1}+.....+a_{n-1}s+a_n = 0
\end{multline}
the Routh array can be constructed as 
 
\begin{align}
\mydet{s^n\\s^{n-1}\\s^{n-2} \\ \vdots}
 \mydet{a_0 & a_2 & a_4 & \cdots \\
a_1 & a_3 & a_5 & \cdots  \\
b_1 & b_2 & b_3 & \cdots \\
\vdots & \vdots & \vdots & \ddots &\vdots 
 \cdots \\}
\end{align}
%
 where
 \begin{align}
 b_1 =\frac{ a_1a_2-a_0a_3}{a_1}  
 \\
 b_2 =\frac{ a_1a_4-a_0a_5}{a_1} 
 \\
 c_1=\frac{ b_1a_3-a_1b_2}{b_1} 
\\
 c_2=\frac{ b_1a_5-a_1b_3}{b_1}  
\end{align}
For poles to lie on imaginary axis any one entire row of hurwitz matrix should be zero.
Constructing the routh array for the characteristic equation obtained in \ref{eq:routh_char_eq},
%
\begin{align}
 s^3+3s^2+2s+k = 0
\end{align}
%
\begin{align}
\mydet{s^3\\s^2\\s^1 \\ s^0}
\mydet{1 & 2 \\ 3 & k \\  \frac{6-k}{3} & 0\\ k & 0}
\end{align}
For poles on $\j \omega$ axis any one of the row should be zero.
%
\begin{align}
\therefore \frac{6-k}{3} &= 0 \text{ or } k = 0
\\
\implies k &= 6 \quad \because k > 0
\end{align}
\item Repeat the above using the determinant method.
\\
\solution The {\em Routh matrix} can be expressed as
\begin{align}
\vec{R} = \myvec{a_0 & a_2 & a_4 & \cdots \\
a_1 & a_3 & a_5 & \cdots  \\
 0 & a_0 & a_2\cdots \\
 0 & a_1 & a_3 \cdots\\
\vdots & \vdots & \vdots & \ddots &\vdots 
\cdots \\}
\end{align}
and the corresponding Routh determinants are
\begin{align}
D_1 &= |a_0|
\\
D_2 &= 
\mydet{
a_0 & a_2 
\\ 
a_1 & a_3
} 
\\
D_3 &=\mydet{
a_0 & a_2 & a_4 
\\ a_1 & a_3 & a_5 
\\ 0 & a_0 & a_2}
\\
\dots
\end{align}
If at least any one of the Determinents are zero then the poles lie on imaginary axes.  From \eqref{eq:routh_char_eq},
%
\begin{align}
D_1 &= 1 \ne 0
\\
D2 &= \mydet{
1 & 2 \\ 3 & k } 
&= k-6 =0 \implies k = 6
\end{align}
%
\item Verify your answer using a python code for both the determinant method as well as the tabular method.
\\
\solution 
The following code 
%
\begin{lstlisting}
codes/ee18btech11005/ee18btech11005.py
\end{lstlisting}
%
provides the necessary soution.
\begin{itemize}
\item  For the system to be stable all coefficients should lie on left half of s-plane. Because if any pole is in right half of s-plane then there will be a component in output that increases without bound,causing system to be unstable.
All the coefficients in the characteristic equation should be positive.This is necessary condition but not sufficient.Because it may have poles on right half of s plane.
Poles are the roots of the characteristic equation.
    \item A system is stable if all of its characteristic modes go to finite value as t goes to infinity.It is possible only if all the poles are on the left half of s plane.
    The characteristic equation should have negative roots only. So the first column should always be greater than zero.That means no sign changes.
    \item A system is unstable if its characteristic modes are not bounded. Then the characteristic equation will also have roots in the right side of s-plane.That means it has sign changes.
    \end{itemize}

\end{enumerate}


}
	\end{center}
\caption{}
\label{fig:ee18btech11005}
\end{figure}

\solution $\because H(s) = 1$ in Fig.  \ref{fig:ee18btech11005}, due to unity feedback,   the transfer function is given by
\begin{align}
\frac{Y(s)}{X(s)} &= \frac{G(s)}{1+G(s)H(s)}
\\
\implies T(s) &= \frac{k}{s^3+3s^2+2s}
\end{align}
%
\item Find the {\em characteristic} equation for $G(s)$.
\\
\solution The characteristic equation is
\begin{align}
\label{eq:routh_char_eq}
 1 + G(s)H(s) &= 0 
\\
\implies 1 + \sbrak{\frac{k}{s^3+3s^2+2s}} &= 0
\\
\text{or, } s^3+3s^2+2s+k &= 0
\end{align}
\item Using the tabular method for the Routh hurwitz criterion, find $k > 0$ for which there are two poles of unity feedback system on j${\omega}$ axis.
%
\\
\solution 
This criterion is based on arranging the coefficients of characteristic equation into an array called Routh array.
For any characteristic equation 
\begin{multline}
q(s) = a_0s^n+a_1s^{n-1}+.....+a_{n-1}s+a_n = 0
\end{multline}
the Routh array can be constructed as 
 
\begin{align}
\mydet{s^n\\s^{n-1}\\s^{n-2} \\ \vdots}
 \mydet{a_0 & a_2 & a_4 & \cdots \\
a_1 & a_3 & a_5 & \cdots  \\
b_1 & b_2 & b_3 & \cdots \\
\vdots & \vdots & \vdots & \ddots &\vdots 
 \cdots \\}
\end{align}
%
 where
 \begin{align}
 b_1 =\frac{ a_1a_2-a_0a_3}{a_1}  
 \\
 b_2 =\frac{ a_1a_4-a_0a_5}{a_1} 
 \\
 c_1=\frac{ b_1a_3-a_1b_2}{b_1} 
\\
 c_2=\frac{ b_1a_5-a_1b_3}{b_1}  
\end{align}
For poles to lie on imaginary axis any one entire row of hurwitz matrix should be zero.
Constructing the routh array for the characteristic equation obtained in \ref{eq:routh_char_eq},
%
\begin{align}
 s^3+3s^2+2s+k = 0
\end{align}
%
\begin{align}
\mydet{s^3\\s^2\\s^1 \\ s^0}
\mydet{1 & 2 \\ 3 & k \\  \frac{6-k}{3} & 0\\ k & 0}
\end{align}
For poles on $\j \omega$ axis any one of the row should be zero.
%
\begin{align}
\therefore \frac{6-k}{3} &= 0 \text{ or } k = 0
\\
\implies k &= 6 \quad \because k > 0
\end{align}
\item Repeat the above using the determinant method.
\\
\solution The {\em Routh matrix} can be expressed as
\begin{align}
\vec{R} = \myvec{a_0 & a_2 & a_4 & \cdots \\
a_1 & a_3 & a_5 & \cdots  \\
 0 & a_0 & a_2\cdots \\
 0 & a_1 & a_3 \cdots\\
\vdots & \vdots & \vdots & \ddots &\vdots 
\cdots \\}
\end{align}
and the corresponding Routh determinants are
\begin{align}
D_1 &= |a_0|
\\
D_2 &= 
\mydet{
a_0 & a_2 
\\ 
a_1 & a_3
} 
\\
D_3 &=\mydet{
a_0 & a_2 & a_4 
\\ a_1 & a_3 & a_5 
\\ 0 & a_0 & a_2}
\\
\dots
\end{align}
If at least any one of the Determinents are zero then the poles lie on imaginary axes.  From \eqref{eq:routh_char_eq},
%
\begin{align}
D_1 &= 1 \ne 0
\\
D2 &= \mydet{
1 & 2 \\ 3 & k } 
&= k-6 =0 \implies k = 6
\end{align}
%
\item Verify your answer using a python code for both the determinant method as well as the tabular method.
\label{prob:ee18btech11005_python}
\\
\solution 
The following code 
%
\begin{lstlisting}
codes/ee18btech11005.py
\end{lstlisting}
%
provides the necessary soution.
\begin{itemize}
\item  For the system to be stable all coefficients should lie on left half of s-plane. Because if any pole is in right half of s-plane then there will be a component in output that increases without bound,causing system to be unstable.
All the coefficients in the characteristic equation should be positive.This is necessary condition but not sufficient.Because it may have poles on right half of s plane.
Poles are the roots of the characteristic equation.
    \item A system is stable if all of its characteristic modes go to finite value as t goes to infinity.It is possible only if all the poles are on the left half of s plane.
    The characteristic equation should have negative roots only. So the first column should always be greater than zero.That means no sign changes.
    \item A system is unstable if its characteristic modes are not bounded. Then the characteristic equation will also have roots in the right side of s-plane.That means it has sign changes.
    \end{itemize}

\end{enumerate}


